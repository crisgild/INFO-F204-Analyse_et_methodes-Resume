\section{Meta Model UML}



\subsection{Introduction}
Tout les composants qu’on peut utiliser en UML ont été écrits en UML.
\\Un méta-model est un modèle qui en décrit un autre.
\\Chaque diagramme est une instance du méta-model.



\subsection{L'élément}
Tout ce qu’on manipule en UML est un élément.
\\Chaque élément peut être contenu dans un paquet.



\subsection{Mécanismes}
\begin{itemize}
	\item [\textbf{Stéréotype}] permet d’étendre UML. ( format : << stéréotype >>). On veut par exemple montrer que certaines classe servent à la gestions des freins, on leur ajoute le stéréotype <<gestion des freins>>.
	\item [\textbf{Tagged values}] est une association entre un nom et une valeur. (Exemple: créateur: machin)
	\item [\textbf{Notes}] donne des bouts de texte, des commentaires.
	\item [\textbf{Contraintes}] entouré par des accolades. Permet de représenter des contraintes sur les attributs, valeurs, associations, …
	\item [\textbf{Dépendances}] dit qu’un élément dépend d’un autre (flèche en pointillé)
	\item [\textbf{Types prédéfinis}] sont les types qu’on connaît (bool, string, int, …)
	\item [\textbf{Multiplicité}] représente un nombre lors des associations.
	\item [\textbf{Package}] rassemble des éléments.
\end{itemize}



\subsection{Diagrammes des classes}
On y représente les classes utilisées dans le projet. Représente les différentes informations que l’application va devoir manipuler.



\subsubsection{Représentation d’une classe}
Représenté par un grand rectangle contenant en haut le nom de la classe, puis les attributs, puis les méthodes.



\subsubsection{Règles de représentation}
Classe commence par une majuscule.
\\Méthode doit être un nom, non une action ou un verbe.
\\Attribut commence par une minuscule et ont un type ou une visibilité et peut avoir une valeur par défaut.
\\Méthodes ont une signature ( type de retour, le nom, 0 ou plusieurs paramètres) et une visibilité.



\subsubsection{Relations de base}
4 types de relations/associations entre les classes.
\begin{description}
	\item [Usage] Utilisation
	\item [Inheritance] L’héritage
	\item [Refinement] La même classe avec plus de détails.
	\item [Réalisation] Les classes abstraites qu’on réalise.
\end{description}

Pour la relation d’usage on peut ( pas obligatoire) avoir une direction et une multiplicité.

\textbf{Relations ternaires}: Relations entre trois classes.
\textbf{Role name}: Le nom du rôle joué par la classe dans la relation (Exemple: maître ou esclave.)



\subsubsection{Relations étendues}
\begin{itemize}
	\item [Association] Une association représente une relation quelconque entre deux classes ; en général, les objets d’une classe se servant de ceux d’une autre classe. (Exemple : une personne utilise un ordinateur.) Un simple trait entre les deux classes, peut être composé d’une flèche pour indiquer une direction.

	\item [Agrégation]: Une agrégation permet de définir une entité comme étant liée à plusieurs entités de classe différente ; décrit une association de type « fait partie de », « a ». (Exemple: Une flotte constituée de plusieurs bateau. Une flotte n’est plus une flotte sans bateau.) Représenté par un diamant vide du coté de la classe qui agrège (ici la flotte). 

	\item [Composition]: Comme l’agrégation mais en plus fort. Elle fait partie entière de l’objet. Si on détruit le tout, on détruit tous les éléments. (Exemple: Si on détruit le livre, les pages n’ont plus de raison d’être, contrairement aux bateaux sans la flotte.) Représenté par un diamant noir du coté de la classe principale (ici le livre).

	\item [Généralisation]: Héritage, flèche pointant sur la classe parente.

	\item [Raffinement]: Relation d’une classe vers elle même. On représente ça avec << refine>>

	\item [Réalisation]: Dit qu’une classe en réalise une autre.
\end{itemize}