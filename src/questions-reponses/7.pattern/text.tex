\section{Pattern}



\subsection{Question 3 points}



\subsubsection{Expliquez ce qu'est un design pattern}
\textcolor[rgb]{0,0.48,0.58}{Ils permettent de remplir le fossé entre " je comprends l'orienté objet " et " je réussis à bien l'utiliser ". Ce sont des descriptions de classes communiquant entre eux et qui ont été modifiées afin de résoudre des problèmes généraux, dans un contexte particulier.
\\Ils ne permettent pas :
\begin{itemize}
	\item D'obtenir une solution à un problème d'implémentation.
	\item D'avoir une solution toute faite et prête à être utilisée.
\end{itemize}
Car ils sont beaucoup trop génériques pour ça.
\\Leur but est donc principalement :
\begin{itemize}
	\item D'améliorer sa maîtrise/son contrôle de l'orienté objet.
	\item De mieux se comprendre/communiquer grâce à un vocabulaire précis.
\end{itemize}
Un pattern est composé :
\begin{itemize}
	\item D'un nom.
	\item D'une description du problème qu'il veut résoudre.
	\item De la description de la solution.
	\item De sa conséquence.
\end{itemize}
Il y a 23 patterns de base et on peut les combiner.
\\Il y a 3 grands types de designs patterns :
\begin{itemize}
	\item Les \textbf{creationnals patterns} : pattern dont le but est de créer des objets de manière propre : en limitant leurs nombre (via singleton), leur couplage...
	\item Les \textbf{structurals patterns} : patterns qui permettent de résoudre les problèmes structurels. Séparation entre l'interface et l'implémentation,…
	\item Les \textbf{behaviours patterns} : patterns qui permettent de résoudre les problèmes au niveau du comportement, des algorithmes, des responsabilités,…
\end{itemize}}



\subsubsection{Énumérez deux avantages de l'utilisation des design patterns dans un projet}
\textcolor[rgb]{0,0.48,0.58}{Communication simplifiée : il suffit de dire " j’ai utilisé le pattern X ici " pour que l’interlocuteur comprenne la structure de cette partie du système, sans devoir tout expliquer dans le détail.
\\On est assurés que la solution fonctionne et fonctionne bien, elle a déjà été testée souvent.}



\subsubsection{Présentez et expliquez l’utilité du pattern observateur (Observer pattern)}
\textcolor[rgb]{0,0.48,0.58}{Il définit une relation entre plusieurs objets de manière à ce que lorsqu’un objet change d’état, les objets dépendants sont automatiquement mis à jour.}